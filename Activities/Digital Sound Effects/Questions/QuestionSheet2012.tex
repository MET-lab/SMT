\documentclass[11pt]{article}
\usepackage{epsfig}
\usepackage{alltt}
\usepackage{amsmath}
\usepackage{multicol}
\usepackage{wrapfig}

\newcommand{\ben}{\begin{enumerate}}
\newcommand{\een}{\end{enumerate}}
\newcommand{\bin}{\begin{itemize}}
\newcommand{\ein}{\end{itemize}}
\newcommand{\myTilde}{\raise.17ex\hbox{$\scriptstyle\mathtt{\sim}$}}
\newcommand{\ohm}{$\Omega$ }
\newcommand{\abs}[1]{\lvert#1\rvert}

\marginparwidth 0pt
\oddsidemargin  0pt
\evensidemargin  0pt
\marginparsep 0pt
\parindent 0pt
\parskip 7.2pt

\topmargin   -0.5in

\textwidth   6.5in
\textheight  9 in


%%% BEGIN DOCUMENT
\begin{document}

{\bf \Large Name } \line(1,0) {200}

% Activity 1 ======================================================================================================================
\section*{\bf Activity 1.1 Questions}

In GarageBand...

\begin{enumerate}

\item With just one copy track, when you change the volume of the copy track, how does this change the echo sound you hear? \\

\vspace{50mm}

\item With the original and one copy track, what combinations of time-offset and volume differences between the original and the copy sound the most like real echoes?  What settings sound like unnatural echoes?  Draw pictures of the settings if it helps.

\vspace{50mm}

\item When you add a third track, what settings would you use to make it sound more natural, and why do you think this is so?

\end{enumerate}

\newpage

% Activity 2 ======================================================================================================================
\section*{Activity 1.2 Questions}

In Dingsaller...

\begin{enumerate}

\item For a \emph{single} echo (like that shown in Figure 5 of the handout), when experimenting with very short time delay settings (shorter than two distinct hits), did you notice a change in the sound compared to longer delay times?  If so, describe the change.

\vspace{40mm}

\item Still working with a single echo, how would you set the feedback slider to obtain the same effect as the single echo we modeled in GarageBand?

\vspace{30mm}

\item Think about that relates the distance ${\bf d}$ that sound travels over a time ${\bf t}$.  \[d = v\times t\] where ${\bf v}$ is the speed of sound \(({\bf v} \approx 340 \frac{\text{meters}}{\text{second}})\).  Using this equation, compute the following distance: \\

You're standing at the Snake River Corssing at the Grand Canyon.  Yelling from one wall, you hear a single echo bounce back off a wall far away.  If the echo is heard 2 seconds later, how far away is the wall that created the echo?  Draw a picture of the physical environment to help you figure it out.  

\newpage

\vspace{50mm}

\item Working with \emph{two} echoes and zero feedback, set the delay times with a two-to-one relationship (e.g 0.2 seconds and 0.4 seconds) and listen.  Then set the delay times to an unrelated pair of numbers (e.g. 0.31 and 0.5).  Write down the times you used.  Which sounds better to you?

\vspace{50mm}

\item What do you think happens when you add feedback to the delay effect?  Give your best explanation in terms of what the feedback does to the input signal and the delayed copies.  \\ 

\vspace{1mm}

{\bf Hint}:  Think about the feedback that occurs when you hold a microphone up to a PA speaker.  The signal from the speaker is being fed back into the amplifier in a loop, so an amplified output is getting fed back to the input and amplified even more, getting louder each time.  Compare this to what happens when you set the {\bf MultiTap Delay} feedback slider to its maximum value, and note what happens when you decrease the value of the feedback slider.  The parameter that the feedback slider is controlling has something to do with the rate at which the delayed copies are getting quieter, and why the delayed copies sometimes get louder.

\end{enumerate}

\end{document}













